% Notes and exercises from The Lambda Calculus by Barendregt
% By John Peloquin
\documentclass[letterpaper,12pt]{article}
\usepackage{amsmath,amssymb,amsthm,enumitem,fourier}

\newcommand{\lam}{\lambda}
\newcommand{\Lam}{\Lambda}

\newcommand{\inc}{\mathrel{\sharp}}
\newcommand{\sub}{\mapsto}

\newcommand{\comb}[1]{\mathbf{#1}}
\newcommand{\I}{\comb{I}}
\newcommand{\K}{\comb{K}}
\renewcommand{\S}{\comb{S}}
\newcommand{\Y}{\comb{Y}}

\DeclareMathOperator{\FV}{FV}

% Theorems
\theoremstyle{definition}
\newtheorem*{exer}{Exercise}

\theoremstyle{remark}
\newtheorem*{rmk}{Remark}

% Meta
\title{Notes and exercises from\\\textit{The Lambda Calculus}}
\author{John Peloquin}
\date{}

\begin{document}
\maketitle

\section*{Introduction}
This document contains notes and exercises from~\cite{barendregt}.

\section*{Chapter~2}

\begin{exer}[2.4.1]
The following terms have normal forms:
\begin{enumerate}[itemsep=0pt]
\item[(i)] \((\lam y.yyy)((\lam ab.a)\I(\S\S))\)
\item[(ii)] \((\lam yz.zy)((\lam x.xxx)(\lam x.xxx))(\lam w.\I)\)
\end{enumerate}
\end{exer}
\begin{proof}\
\begin{enumerate}[itemsep=0pt]
\item[(i)] \((\lam y.yyy)((\lam ab.a)\I(\S\S))=(\lam y.yyy)\I=\I\I\I=\I\)
\item[(ii)] \((\lam yz.zy)((\lam x.xxx)(\lam x.xxx))(\lam w.\I)=(\lam w.\I)((\lam x.xxx)(\lam x.xxx))=\I\)\qedhere
\end{enumerate}
\end{proof}

\begin{exer}[2.4.2]
The following terms are incompatible:
\begin{enumerate}[itemsep=0pt]
\item[(i)] \(\I\inc\K\)
\item[(ii)] \(\I\inc\S\)
\item[(iii)] \(xy\inc xx\) (\(x\not\equiv y\))
\end{enumerate}
\end{exer}
\begin{proof}
By reducing to \(\K\inc\S\) (2.1.33).
\begin{enumerate}[itemsep=0pt]
\item[(i)] If \(\I=\K\), then \(\S=\I\S=\I\I\S=\K\K\S=\K\).
\item[(ii)] If \(\I=\S\), then since \(\I=\S\K\K\) (2.2.1(i)), \(\S=\I\S=\S\K\K\S=\I\K\K\S=\K\K\S=\K\).
\item[(iii)] If \(xy=xx\), then \(MN=MM\) for all \(M,N\in\Lam\) by rule~\(\xi\) and combinatory completeness (2.1.24). Therefore \(\S=\I\S=\I\I=\I\), which reduces to~(ii).\qedhere
\end{enumerate}
\end{proof}

\begin{exer}[2.4.4]
Application is not associative; in fact, \((xy)z\inc x(yz)\) for distinct \(x,y,z\).
\end{exer}
\begin{proof}
If \((xy)z=x(yz)\), then \((MN)P=M(NP)\) for all \(M,N,P\in\Lam\) by rule~\(\xi\) and combinatory completeness (2.1.24). In particular,
\[\S=\I\S=((\K\I)\K)\S=(\K(\I\K))\S=\K\K\S=\K\]
The result follows since \(\K\inc\S\) (2.1.33).
\end{proof}

\begin{exer}[2.4.6]
There is no \(F\in\Lam\) such that \(F(MN)=M\) for all \(M,N\in\Lam\).
\end{exer}
\begin{proof}
If there were such an~\(F\), then taking \(M=\I\) and \(N=\Y F\) (2.1.5),\footnote{Here \(\Y F\)~denotes the fixed point of~\(F\) in~2.1.5.}
\[\I=F(\I(\Y F))=F(\Y F)=\Y F\]
But this contradicts the fact that \(\Y F\)~has no normal form.
\end{proof}

\begin{exer}[2.4.7]
There is \(M\in\Lam\) such that \(MN=MM\) for all \(N\in\Lam\).
\end{exer}
\begin{proof}
This is true if \(M=\lam x.MM\), which is true if \(M=(\lam yx.yy)M\). So we can take \(M=\Y(\lam yx.yy)\) (2.1.5).
\end{proof}

\begin{exer}[2.4.9]
\((\lam y.(\lam x.M))N=\lam x.((\lam y.M)N)\) for all \(M,N\in\Lam\), provided \(x\not\equiv y\) and \(x\not\in\FV(N)\).
\end{exer}
\begin{proof}
By \(\beta\)-reduction,
\[(\lam y.(\lam x.M))N=(\lam x.M)[y\sub N]=\lam x.M[y\sub N]=\lam x.((\lam y.M)N)\qedhere\]
\end{proof}
\begin{rmk}
The hypotheses are necessary. For example if \(x,y,z\) are distinct,
\[(\lam x.(\lam x.x))y=\lam x.x\ne\lam x.y=\lam x.((\lam x.x)y)\]
and
\[(\lam y.(\lam x.y))x=\lam z.x\ne\lam x.x=\lam x.((\lam y.y)x)\]
\end{rmk}

\begin{exer}[2.4.13] \(\lam x.Mx=M\) for all \(M\in\Lam\) starting with~\(\lam\), provided \(x\not\in\FV(M)\).
\end{exer}
\begin{proof}
If \(M\equiv\lam y.N\), then
\[\lam x.(\lam y.N)x=\lam x.N[y\sub x]\equiv\lam y.N\equiv M\]
Note this is justified when \(y\not\equiv x\) because \(x\not\in\FV(N)\), so the only free occurrences of~\(x\) in~\(N[y\sub x]\) correspond to free occurrences of~\(y\) in~\(N\).
\end{proof}
\begin{rmk}
The hypotheses are necessary. For example, \(\lam x.yx\ne x\) and
\[\lam x.(\lam y.x)x=\lam x.x\ne\lam y.x\]
\end{rmk}

\begin{exer}[2.4.14]
\(M\equiv\lam x.x(\lam y.yy)(\lam y.yy)\) is \(I\)-solvable.
\end{exer}
\begin{proof}
\(M(\K(\K\I))=\I\).
\end{proof}

\begin{exer}[2.4.15]
We can write down short \(\lambda\)-terms having very long normal forms using the exponentiation combinator on Church numerals (2.2.6).
\end{exer}

% References
\begin{thebibliography}{0}
\bibitem{barendregt} Barendregt, H. \textit{The Lambda Calculus, its Syntax and Semantics.} 2012.
\end{thebibliography}
\end{document}
